% tipo por omisión; no se requiere nada más
\documentclass[]{article}

% layout
\usepackage[letterpaper, margin = 1.5cm]{geometry}

% entornos (alinea ecuaciones, referencias internas)
\usepackage{amsmath}

% manejo de posición de tablas
\usepackage{float}

\begin{document}
    \title {
        Organización y Arquitectura de Computadoras \\
        2019-2 \\
        Desempeño
    }
    \author{
        Edgar Quiroz Castañeda
    }
    \date{
        25 de febrero del 2019
    }
    \maketitle

    \section{
        Ecuación de desempeño del CPU
    }
    Calcula el tiempo de ejecución de cada programa en diferentes computadoras.\\
    En general, con $f$ la frecuencia, $c$ la cantidad de ciclos, y $t$ el 
    tiempo, se tiene que
    \begin{align}
        &f_{K} = \frac{c_{K}}{t_{K}} \nonumber \\
        &\implies t_{K} = \frac{c_{K}}{f_{K}} \label{eq1}
    \end{align}
    Y con $\lambda$ la longitud de un ciclo (inversa de la frecuencia), tenemos 
    que 
    \begin{align}
        &\lambda_{K} = \frac{t_{K}}{c_{K}} \nonumber \\
        &\implies t_{K} = t_{K} \cdot \lambda_{K} \label{eq2}
    \end{align}
    En el caso de $A$ (2.7 MGHz) y $C$ (48 MGHz), se conoce la frecuencia, por 
    lo que se va a utilizar
    \eqref{eq1}. 
    \[
    \vec{t}_{A, C} = 
    \begin{bmatrix}
        24 \times 10^6 & 24 \times 10^6\\ 
        28 \times 10^6 & 28 \times 10^6\\
        15 \times 10^6 & 15 \times 10^6\\
        300 \times 10^3 & 300 \times 10^3\\
        100 \times 10^6 & 100 \times 10^6
    \end{bmatrix}    
    \cdot
    \begin{bmatrix}
        \frac{1}{2.7 \times 10^{9}} & 0\\
        0 & \frac{1}{48 \times 10^{6}}
    \end{bmatrix}
    \approx
    \begin{bmatrix}
        0.00888 & 0.5 \\
        0.01778 & 1 \\
        0.00556 & 0.3125 \\
        0.000112 & 0.00625 \\ 
        0.03704 & 2.083
    \end{bmatrix}
    \]

    Para $B$ (3ns/ciclo) y $D$ (4ms/ciclo), se usará \eqref{eq2}, pues se conoce 
    su longitud de ciclo.

    \[
    \vec{t}_{B,D} = 
    \begin{bmatrix}
        24 \times 10^6 & 24 \times 10^6\\ 
        28 \times 10^6 & 28 \times 10^6\\
        15 \times 10^6 & 15 \times 10^6\\
        300 \times 10^3 & 300 \times 10^3\\
        100 \times 10^6 & 100 \times 10^6
    \end{bmatrix}    
    \begin{bmatrix}
        3 \times 10^{-9} & 0\\
        0 & 4 \times 10^{-3}
    \end{bmatrix}
    \approx
    \begin{bmatrix}
        0.072 & 96000 \\
        0.144 & 192000 \\
        0.045 & 60000 \\
        0.0009 & 1200 \\
        0.3 & 400000
    \end{bmatrix}
    \]

    Por lo que la tabla de tiempos completa es 
    \begin{table}[!h]
        \centering
        \begin{tabular}{|c|c|c|c|c|}
            \hline
            Ciclos & $A$ & $B$ & $C$ & $D$ \\
            \hline
            24 M & 0.00888 seg & 0.072 seg & 0.5 seg & 96000 seg \\
            28 M & 0.01778 seg & 0.144 seg & 1 seg & 192000 seg \\
            15 M & 0.00556 seg & 0.045 seg & 0.3125 seg & 60000 seg \\
            300 K & 0.000112 & 0.0009 seg & 0.00625 seg & 1200 seg \\
            100 M & 0.03704 seg & 0.3 seg & 2.083 seg & 400000 seg\\
            \hline
        \end{tabular}
    \end{table}

    \section{
        Medidas de tendencia central
    }

    Considera los siguientes tiempos de ejecución

    \begin{table}[H]
        \centering
        \caption{Datos}
        \begin{tabular}{|c|c|c|c|c|}
            \hline
             & $A$ & $B$ & $C$ & $D$ \\
            \hline
            Prog 1 & 83ns & 20ns & 56ns & 121ns \\
            Prog 2 & 1938ns & 949ns & 1453ns & 2342ns \\
            Prog 3 & 844ns & 939ns & 932ns & 638ns \\
            Prog 4 & 994ns & 593ns & 734ns & 1029ns \\
            Prog 5 & 95ns & 19ns & 24ns & 68ns \\
            \hline
        \end{tabular}
    \end{table}
    Encuentre las medias aritmética, armónica y geométrica.

    \begin{enumerate}
        \item {
            Para la media aritmética, tenemos que 
            \[\vec{M}_{arit} = 
            \frac{1}{5}
            \begin{bmatrix}
                1 & 1 & 1 & 1 & 1 
            \end{bmatrix}
            \begin{bmatrix}
                83 & 20 & 56 & 121 \\
                1938 & 949 & 1453 & 2342 \\
                844 & 939 & 932 & 638 \\
                994 & 593 & 734 & 1029 \\
                95 & 19 & 24 & 68 \\
            \end{bmatrix}
            = 
            \frac{1}{5}
            \begin{bmatrix}
                3954 & 2520 & 3199 & 4198
            \end{bmatrix}
            = 
            \begin{bmatrix}
                790.8 & 504.  & 639.8 & 839.6
            \end{bmatrix}
            \]
        }
        \item {
            Para la media armónica tenemos que 
            \[\vec{M}_{arm}^{-1} = 
            \frac{1}{5}
            \begin{bmatrix}
                1 & 1 & 1 & 1 & 1 
            \end{bmatrix}
            \begin{bmatrix}
                \frac{1}{ 83} &\frac{1}{ 20} &\frac{1}{ 56} & \frac{1}{121} \\
                \frac{1}{938} & \frac{1}{949} & \frac{1}{453} & \frac{1}{2342} \\
                \frac{1}{844} & \frac{1}{939} & \frac{1}{932} & \frac{1}{638} \\
                \frac{1}{994} & \frac{1}{593} & \frac{1}{734} & \frac{1}{1029} \\
                \frac{1}{ 95} &\frac{1}{ 19} &\frac{1}{ 24} & \frac{1}{68} \\
            \end{bmatrix}
            \approx
            \frac{1}{5}
            \begin{bmatrix}
                0.02528 & 0.10644 & 0.06265 & 0.02594
            \end{bmatrix}
            \]
            Por lo que 
            \[\vec{M}_{arm} \approx
            5 
            \begin{bmatrix}
                39.554811 & 9.39526 & 15.96235 & 38.55563
            \end{bmatrix}
            \approx 
            \begin{bmatrix}
                192.77405 & 46.9763  & 79.81177 & 192.77817
            \end{bmatrix}
            \]
        }
        \item {
            Y para la media geométrica
            \begin{align*}
                \vec{M}_{geo} ^{5}
                &= 
                \begin{bmatrix}
                    83 \cdot 1938\cdot  844\cdot 994 \cdot 95 
                    & 20 \cdot 949 \cdot 939 \cdot 593 \cdot 19  
                    & 56 \cdot 1453 \cdot 932 \cdot 734 \cdot 24
                    & 121 \cdot 2342 \cdot 638 \cdot 1029 \cdot 68
                \end{bmatrix} \\
                &= 
                \begin{bmatrix}
                    12819890077680 & 200802952740 & 1335908937216 & 12650777783952
                \end{bmatrix}
            \end{align*}
            Por lo que 
            \[\vec{M}_{geo} \approx 
            \begin{bmatrix}
                418.38572 & 182.20236 & 266.16771 & 417.27603
            \end{bmatrix}
            \]
        }
    \end{enumerate}
    Entonces, la tabla completa de medias es 
    \begin{table}[H]
        \centering
        \caption{Medias}
        \begin{tabular}{|c|c|c|c|c|}
            \hline
             Medias & $A$ & $B$ & $C$ & $D$ \\
            \hline
            Aritmética & 790.8 & 504 & 639.8 & 839.6 \\
            Armónica & 197.77405481  & 46.97631186 & 79.81177196 & 192.77817439 \\
            Geométrica & 418.38572689 & 182.20236881 & 266.1677118 & 417.27603733 \\
            \hline
        \end{tabular}
    \end{table}
    
    Normaliza $A$ y $D$ respecto a $B$.
    \[\hat{AD}_{B} = 
    \begin{bmatrix}
        83 / 20 & 121 / 20 \\
        1938 / 949 & 2342 / 949 \\
        844 / 939 & 638 / 939 \\
        994 / 593 & 102 / 5939 \\
        95 / 19 & 68 / 19 \\ 
    \end{bmatrix}
    \approx 
    \begin{bmatrix}
        4.15 & 6.05 \\
        2.04214963 & 2.46786091 \\
        0.89882854 & 0.67944622 \\
        1.6762226 & 1.73524452 \\
        5 & 3.57894737 \\
    \end{bmatrix}
    \]
    Por lo que la tabla de datos normalizados es
    \begin{table}[H]
        \centering
        \caption{Datos normalizados respecto a B}
        \begin{tabular}{|c|c|c|c|c|}
            \hline
             & $A$ & $D$ \\
            \hline
            Prog 1 & 4.15 & 6.05 \\
            Prog 2 & 2.04214963 & 2.46786091 \\
            Prog 3 & 0.89882854 & 0.67944622 \\
            Prog 4 & 1.6762226 & 1.73524452 \\
            Prog 5 & 5 & 3.57894737 \\
            \hline
        \end{tabular}
    \end{table}

    Encuetra las medias de los datos normalizados

    \begin{enumerate}
        \item {
            Para la media aritmética, tenemos que 
            \[\vec{M}_{arit} = 
            \frac{1}{5}
            \begin{bmatrix}
                1 & 1 & 1 & 1 & 1
            \end{bmatrix}
            \begin{bmatrix}
                4.15 & 6.05 \\
                2.04214963 & 2.46786091 \\
                0.89882854 & 0.67944622 \\
                1.6762226 & 1.73524452 \\
                5 & 3.57894737 \\
            \end{bmatrix}
            \approx
            \frac{1}{5}
            \begin{bmatrix}
                13.76720 & 14.511499
            \end{bmatrix}
            = 
            \begin{bmatrix}
                2.75344 & 2.90229
            \end{bmatrix}
            \]
        }
        \item {
            Para la media armónica tenemos que 
            \[\vec{M}_{arm}^{-1} = 
            \frac{1}{5}
            \begin{bmatrix}
                1 & 1 & 1 & 1 & 1
            \end{bmatrix}
            \begin{bmatrix}
                \frac{1}{4.15} & \frac{1}{6.05} \\
                \frac{1}{2.04214963} & \frac{1}{2.46786091} \\
                \frac{1}{0.89882854} & \frac{1}{0.67944622} \\
                \frac{1}{1.6762226} & \frac{1}{1.73524452} \\
                \frac{1}{5} & \frac{1}{3.57894737} \\
            \end{bmatrix}
            \approx
            \frac{1}{5}
            \begin{bmatrix}
                2.63978 & 2.89798
            \end{bmatrix}
            \]
            Por lo que 
            \[\vec{M}_{arm} \approx
            5 
            \begin{bmatrix}
                0.378819 & 0.345067
            \end{bmatrix}
            \approx 
            \begin{bmatrix}
                1.8940953 & 1.7253369
            \end{bmatrix}
            \]
        }
        \item {
            Y para la media geométrica
            \begin{align*}
                \vec{M}_{geo} ^{5}
                &= 
                \begin{bmatrix}
                    4.15 \cdot 2.04214963 \cdot 0.89882854 \cdot 1.6762226 \cdot 5 
                    & 6.05 \cdot 2.46786091 \cdot 0.67944622 \cdot 1.73524452 \cdot 3.57894737\\
                \end{bmatrix} \\
                &= 
                \begin{bmatrix}
                    63.84313528635815 & 63.00095497267039
                \end{bmatrix}
            \end{align*}
            Por lo que 
            \[\vec{M}_{geo} \approx 
            \begin{bmatrix}
                2.296269 & 2.29017899
            \end{bmatrix}
            \]
        }    
        
    \end{enumerate}
    Por lo que la tabla completa de medidas es 
    \begin{table}[H]
        \centering
        \caption{Medias con datos normalizados}
        \begin{tabular}{|c|c|c|c|c|}
            \hline
             Medias & $A$ & $D$ \\
            \hline
            Aritmética & 2.75344015 & 2.9022998   \\
            Armónica & 1.89409533 & 1.7253369 \\
            Geométrica & 2.29626942 & 2.29017899 \\
            \hline
        \end{tabular}
    \end{table}
    
    Y compara el desempeño usando la media geométrica no normalizada.\\
    Las computadoras, de mejor a peor desempeño, son $B$, $C$, $D$ y $A$.

    \section{
        Ley de Amdahl
    }

    \begin{enumerate}
        \item {
            Una transformación común requerida en los procesadores es la 
            raíz cuadrada. Supongoamos que esta es responsable del 25\% del 
            tiempo total de ejecución. Se tienen dos propuestas para mejorar.
            \begin{enumerate}
                \item {
                    Modificar el hardware de forma que la operación sea 8 
                    veces más rápida.
                }
                \item {
                    Tratar que las operaciones que ocupan el 55\% del 
                    tiempo restante se ejecuten 1.8 veces más rápido.
                }
            \end{enumerate}
            ¿Cómo mejora el desempeño en ambos casos?¿Cuál es mejor? \\
            En el primer caso hay una ganancia de 
            \[G = \frac{1}{(1-0.25) + \frac{0.25}{8}} = \frac{1}{0.75 + 0.03125} 
            = \frac{1}{0.78125} = 1.28\]
            En el segundo caso, la ganancia es de 
            \[G = \frac{1}{(1-0.55) + \frac{0.55}{1.8}} \approx \frac{1}{0.45 + 0.305} 
            = \frac{1}{0.755} \approx 1.324\]
            Por lo que es más conveninte tratar de mejorar el resto de las 
            operaciones.
        }
        \item {
            Tenemos un programa que consiste de 3 procesos $A$, $B$ y $C$. El 
            tiempo que tarda en ejecutarse es de 17 segundos. Los procesos toman
            32\%, 7\% y 25\% del tiempo respectivamente. Si se mejor el proceso
            B 20 veces, ¿cuál es el nuevo tiempo de ejecución?. ¿Y si se mejora
            el proceso A 1.4 veces? ¿Qué mejora es mejor?\\
            En el primer caso, se tiene un tiempo mejorado de 
            \[T_{m} = 17 \cdot ((1-0.07) + \frac{0.07}{20}) 
            = 17 \cdot (0.93 + 0.0035) = 17 \cdot 0.9335 = 15.8695\]
            En el segundo caso, el nuevo tiempo es de 
            \[T_{m} = 17 \cdot ((1-0.32) + \frac{0.32}{1.4}) 
            \approx 17 \cdot (0.68 + 0.22857) = 17 \cdot 0.90867 = 15.4457\]
            Por lo que es más conveninte mejorar lel proceso A 1.4 veces.
        }
        \item {
            Se mejora un proceso que tomaba 70\% de tiempo de ejecución. Se 
            obtubo una ganancia neta de 2.64. ¿Cuál es la ganancia bruta? \\
            La ganancia neta está dada por 
            \[G = \frac{1}{1-F + \frac{F}{g}}\]
            Con $F$ el porcentaje de tiempo y $g$ la ganancia bruta. \\
            Despejando $g$, 
            \begin{align*}
                G &= \frac{1}{1-F + \frac{F}{g}} \\
                \implies 1-F + \frac{F}{g} &= \frac{1}{G} \\
                \implies  \frac{F}{g} &= \frac{1}{G} + F - 1 \\
                \implies g &= \frac{F}{\frac{1}{G} + F - 1}
            \end{align*}
            Por lo que, es este caso, la ganancia bruta es de 
            \[g = \frac{0.7}{\frac{1}{2.64} + 0.7 - 1} 
            \approx \frac{0.7}{0.078788} = 8.885\]
        }
    \end{enumerate}
\end{document}